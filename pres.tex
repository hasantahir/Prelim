\documentclass[mathserif,18pt,xcolor=table]{beamer}

% Load Beamer Style Theme
% TAMU Based
\usepackage{tamu_beamer}
\usepackage[font=small,skip=0pt]{caption}
\usepackage{etoolbox}
% \preto\subequations{\ifhmode\unskip\fi}
% \AtBeginEnvironment{subequations}{\ifhmode\unskip\fi}
% \AtBeginEnvironment{equation}{\ifhmode\unskip\fi}
\makeatletter
\g@addto@macro\normalsize{%
\setlength\abovedisplayskip{0pt}
\setlength\belowdisplayskip{0pt}
\setlength\abovedisplayshortskip{0pt}
\setlength\belowdisplayshortskip{0pt}
}
\makeatother


% Specifiy the location of images to be used
\graphicspath{{figures/}}


% Document Title Page
\title{Sommerfeld Integral}
\subtitle{Horizontally Oriented Magnetic Dipole above Silver Half-plane}
\author[Hasan Tahir Abbas]{ \underline{Hasan~Tahir~Abbas}}
\institute{Department of Electrical  \& Computer Engineering\\ \mbox{} \\ \pgfuseimage{tamuecenbig}}
\date[Spring 2017]{}
% \date[Spring 2017]{\today}

\begin{document}
\preto\subequations{\ifhmode\unskip\fi}
\AtBeginEnvironment{subequations}{\ifhmode\unskip\fi}
\AtBeginEnvironment{equation}{\ifhmode\unskip\fi}
% Draw Boxes in the footer with pertinent info
\tikzstyle{block} = [rectangle, draw, rounded corners, shade, top color=white, text width=5em,
bottom color=blue!50!black!20, draw=blue!40!black!60, very thick, text centered, minimum height=4em]
\tikzstyle{line} = [draw, -latex']
\tikzstyle{cloud} = [draw, ellipse,top color=white, bottom color=red!20, node distance=2cm, minimum height=2em]

% Tick Style
\beamertemplateballitem
% \beamertemplatetransparentcoveredhigh

\frame{\titlepage}


% Add TAMU logo on each slide in the north-east side
% Shifted to be right at the edge
\addtobeamertemplate{frametitle}{}{%
\begin{tikzpicture}[remember picture,overlay]
  \node[anchor=north east,yshift=5pt,xshift=2pt] at (current page.north east) {\includegraphics[height=.7cm]{ecen}};
\end{tikzpicture}}



% Beginning of the actual content
\section{Surface Plasmons in the Terahertz frequency regime}
\begin{frame}
  \frametitle{Two-dimensional Electon Gas (2DEG)}
  \framesubtitle{Introduction}
  \begin{columns} % align columns
    \begin{column}{.5\textwidth}
      \begin{minipage}[T][.1\textheight][c]{\linewidth}
        \begin{itemize}
          \item Semiconductor Heterostructure Interface
          \item High electron Mobility
          \item High concentration of electric charge
          \item \textbf{Surface waves}
          \begin{itemize}
            \item[]{Surface Plasmon-Polaritons}
          \end{itemize}
          \item Formation of Quantum Well
          \begin{itemize}
            \item[]{Two-dimensional confinement of electrons}
          \end{itemize}
        \end{itemize}
      \end{minipage}
      %
    \end{column}
    %
    \begin{column}{.5\textwidth}
      % Use this to preserve fonts from Inkspace
      \begin{figure}
        \vspace*{-1cm}
        \def\svgwidth{\linewidth}
        \input{figures/2deg_bandgap.pdf_tex}
        \caption{Band diagram of a GaAs/AlGaAs heterostructure}
      \end{figure}
      \begin{itemize}
        \item[]{\makebox[.5cm][l]{$E_c$} - Conduction band edge}
        \item[]{\makebox[.5cm][l]{$E_f$} - Fermi level}
      \end{itemize}
      \end{column}%
    \end{columns}
  \end{frame}
  % ------------------------------------------------------------
  % ------------------------------------------------------------
  % ------------------------------------------------------------
  % ------------------------------------------------------------
  % ------------------------------------------------------------
  % ------------------------------------------------------------
  % ------------------------------------------------------------
  \begin{frame}
    \frametitle{Two-dimensional Electon Gas (2DEG)}
    \framesubtitle{Material Description}
    \begin{columns}[T] % align columns
      \begin{column}{.5\textwidth}
        \begin{itemize}
          \item Drude-Lorentz conductivity model
        \end{itemize}
        \begin{subequations} \nonumber
          \begin{align}
            \sigma_s(\O) &= \frac{N e^2 \tau}{m^{\ast}}\frac{\O}{\O + j \tau \left(\O^2 - \O_0^2\right)}
            \label{eq:conductivity}\\
            \O_0 &= \mathrm{\chi}\sqrt{\frac{N e^2}{m^{\ast} \E_{\inf}}}
            \label{eq:plasma_f}
          \end{align}
          \label{eq:eq_currents}
        \end{subequations}
        \begin{itemize}
          \item[]{\makebox[.3cm][l]{$N$} - charge density ($cm^{-3}$)}
          \item[]{\makebox[.3cm][l]{$e$} - electron charge}
          \item[]{\makebox[.3cm][l]{$\tau$} - scattering time}
          \item[]{\makebox[.3cm][l]{$m^{\ast}$} - effective mass of electron}
          \item[]{\makebox[.3cm][l]{$\O_0$} - effective plasma frequency}
        \end{itemize}
      \end{column}
      \begin{column}[T]{.5\textwidth}
        \begin{figure}
          \vspace*{-1cm}
          \includestandalone[width=\linewidth,keepaspectratio]{figures/conductivity_gaas}
          \label{fig:disp}
          % \caption{Conductivity of Gallium Arsenide (GaAs)}
        \end{figure}
        \begin{itemize}
          \item[]{\makebox[.3cm][l]{$\E_{\inf}$} - high frequency limit of dielectric constan}
          \item[]{\makebox[.3cm][l]{$\chi$} - geometrical factor (1/3)}
        \end{itemize}
      \end{column}
    \end{columns}
  \end{frame}
  % ------------------------------------------------------------
  % ------------------------------------------------------------
  % ------------------------------------------------------------
  % ------------------------------------------------------------
  % ------------------------------------------------------------
  % ------------------------------------------------------------
  % ------------------------------------------------------------
  \begin{frame}
    \frametitle{Two-dimensional Electon Gas (2DEG)}
    \framesubtitle{Material Description (contd.)}
    \begin{columns}[T] % align columns
      \begin{column}{.5\textwidth}
        \begin{itemize}
          \item Drude-Lorentz conductivity model
        \end{itemize}
        \begin{equation} \nonumber
          \E(\O) = \E^{\inf} + \prod_i\frac{\O_{li}^2 - \O^2 - j\gamma_{li} \O}{\O_{ti}^2 - \O^2 - j\gamma_{ti} \O}
          \label{eq:eps}
        \end{equation}
        \begin{itemize}
          \item[]{\makebox[.3cm][l]{$\O_{ti}$} - TO phonon frequencies}
          \item[]{\makebox[.3cm][l]{$\O_{li}$} - LO phonon frequencies}
          \item[]{\makebox[.3cm][l]{$\gamma$} - Damping constants}
        \end{itemize}
      \end{column}
      \begin{column}[T]{.5\textwidth}
        % \begin{figure}
        %   \vspace*{-2cm}
        %   \subfloat{\includestandalone[width=.75\linewidth,keepaspectratio]{figures/epsilon_gaas}
        %   \label{fig:eps_Ga}}
        %   \vspace*{0cm}
        %   \subfloat{\includestandalone[width=.75\linewidth,keepaspectratio]{figures/epsilon_sto}
        %   \label{fig:eps_Sto}}
        %   \caption{Dielectric Functions of the materials in bulk form. Solid line: real part, dashed line: imaginary part}
        %   \label{fig:eps}
        % \end{figure}
        \end{column}%
      \end{columns}
    \end{frame}
    % ------------------------------------------------------------
    % ------------------------------------------------------------
    % ------------------------------------------------------------
    % ------------------------------------------------------------
    % ------------------------------------------------------------
    % ------------------------------------------------------------
    % ------------------------------------------------------------
    \begin{frame}
      \frametitle{Two-dimensional Electon Gas (2DEG)}
      \framesubtitle{Dispersion Relation}
      % \begin{itemize}
      %   \item[] Isotropic Environment
      % \end{itemize}
      \begin{columns} % align columns
        \begin{column}[T]{.5\textwidth}
          \begin{itemize}
            \item TE mode
          \end{itemize}
          \begin{equation} \nonumber
            k_{z1} + k_{z2} = \O \sigma_s(\O)
            \label{eq:disp_TE_two}
          \end{equation}
          \begin{itemize}
            \item[] No real solutions for an isotropic environment
          \end{itemize}
          \begin{itemize}
            \item TM mode
          \end{itemize}
          \begin{equation} \nonumber
            \tcbhighmath[drop fuzzy shadow]{\frac{\E_1(\O)}{k_{z1}} + \frac{\E_2(\O)}{k_{z2}} = -\frac{\sigma_s(\O)}{\O}}
            \label{eq:disp_TM_two}
          \end{equation}
          \begin{itemize}
            \item[] Real solution(s). Surface waves exist.
          \end{itemize}
        \end{column}
        %
        \begin{column}[T]{.5\textwidth}
          \begin{equation} \nonumber
            \begin{split}
              k_{zi} & = \sqrt{k_i^2 - k_x^2} \\
              & = \sqrt{\left(\frac{\O}{c}\right)^2 \E_i(\O) -  k_x^2}
            \end{split}
            \label{eq:kz}
          \end{equation}
          % Use this to preserve fonts from Inkspace
          \begin{figure}
            \def\svgwidth{\linewidth}
            \input{figures/2deg.pdf_tex}
            \caption{2DEG at a semiconductor heterojunction}
          \end{figure}
          \end{column}%
        \end{columns}
      \end{frame}
      % ------------------------------------------------------------
      % ------------------------------------------------------------
      % ------------------------------------------------------------
      % ------------------------------------------------------------
      % ------------------------------------------------------------
      % ------------------------------------------------------------
      % ------------------------------------------------------------
      \begin{frame}
        \frametitle{Two-dimensional Electon Gas (2DEG)}
        \framesubtitle{Existence of Surface Waves}
        \begin{columns} % align columns
          \begin{column}[T]{.5\textwidth}
            \begin{equation} \nonumber
              \E_1(\O) \cdot \E_2(\O) < 0
              \label{eq:conditions}
            \end{equation}
            \begin{itemize}
              \item Criterion met at terahertz frequency range
              \item Opposite signs of dielectric constant at Semiconductor interface
              \item GaAs/AlGaAs semiconductor heterostructures
              \item Strontium Titanate/Lanthanum Aluminate (STO/LAO) oxide interfaces
            \end{itemize}
          \end{column}
          \begin{column}[T]{.5\textwidth}
            % \begin{figure}
            %   \vspace*{-2cm}
            %   \subfloat{\includestandalone[width=.75\linewidth,keepaspectratio]{figures/epsilon_gaas}
            %   \label{fig:eps_Ga}}
            %   \vspace*{0cm}
            %   \subfloat{\includestandalone[width=.75\linewidth,keepaspectratio]{figures/epsilon_sto}
            %   \label{fig:eps_Sto}}
            %   \caption{Dielectric Functions of the materials in bulk form. Solid line: real part, dashed line: imaginary part}
            %   \label{fig:eps}
            % \end{figure}
            \end{column}%
          \end{columns}
        \end{frame}
        % ------------------------------------------------------------
        % ------------------------------------------------------------
        % ------------------------------------------------------------
        % ------------------------------------------------------------
        % ------------------------------------------------------------
        % ------------------------------------------------------------
        % ------------------------------------------------------------
        % Beginning of the actual content
        \section{Optical Nanoantennas}
        \begin{frame}
          \frametitle{Optical Nanoantennas}
          \framesubtitle{Introduction}
          \begin{columns} % align columns
            \begin{column}{.5\textwidth}
              \begin{minipage}[T][.1\textheight][c]{\linewidth}
                \begin{itemize}
                  \item Near-field Scanning Electron Microscopy (NSOM)
                  \begin{itemize}
                    \item[] Subwavelength confinement
                  \end{itemize}
                  \item Directivity enhancement of Quantum emitters
                  \item \textbf{Surface waves}
                  \begin{itemize}
                    \item[]{Surface Plasmon-Polaritons}
                  \end{itemize}
                  \item Radiation Mechanism
                  \begin{itemize}
                    \item[]{Wavenumber Mismatch}
                  \end{itemize}
                \end{itemize}
              \end{minipage}
            \end{column}
            %
            \begin{column}{.5\textwidth}
              % Use this to preserve fonts from Inkspace
              % \begin{figure}
              %   \vspace*{-1cm}
              %   \def\svgwidth{\linewidth}
              %   \input{figures/2deg_bandgap.pdf_tex}
              %   \caption{Band diagram of a GaAs/AlGaAs heterostructure}
              % \end{figure}
              \begin{itemize}
                \item[]{\makebox[.5cm][l]{$E_c$} - Conduction band edge}
                \item[]{\makebox[.5cm][l]{$E_f$} - Fermi level}
              \end{itemize}
              \end{column}%
            \end{columns}
          \end{frame}
          % ------------------------------------------------------------
          % ------------------------------------------------------------
          % ------------------------------------------------------------
          % ------------------------------------------------------------
          % ------------------------------------------------------------
          % ------------------------------------------------------------
          % ------------------------------------------------------------
          \begin{frame}
            \frametitle{Optical Nanoantennas}
            \framesubtitle{Dispersion Relation}
            \begin{columns}[T] % align columns
              \begin{column}{.5\textwidth}
                \begin{itemize}
                  \item Metal-dielectric Interface
                \end{itemize}
                \begin{equation} \nonumber
                  k_{sp}=k_{1}\sqrt {\dfrac {\E_{1}\E_{2}(\O)} {\E_{1} + \E_{2}(\O)}}
                  \label{eq:dis_spp}
                \end{equation}
                \begin{itemize}
                  \item Accurate material description using Drude-critical points
                \end{itemize}
                \begin{equation} \nonumber
                    \E_2(\O) = \E_{\inf} - \frac{\O_{d}^{2}}{\O^2 + j\gamma \O} + \sum \limits_{i = 1}^N G_i(\O)
                  \label{eq:eps_drude_cp}
                  \end{equation}
                  \begin{equation} \nonumber
                      G_i(\O) = C_i \left[ \frac{e^(j \phi_i)}{\O_i - \O - j \Gamma_i} + \frac{e^(-j \phi_i)}{\O_i + \O + j \Gamma_i} \right]
                    \label{eq:CP_terms}
                    \end{equation}
              \end{column}
              \begin{column}[T]{.5\textwidth}
                \begin{figure}
                  \vspace*{-2cm}
                  \includestandalone[width=.75\linewidth,keepaspectratio]{figures/ep_gold}
                  \label{fig:ep_gold}
                  % \caption{Conductivity of Gallium Arsenide (GaAs)}
                \end{figure}
                \begin{figure}
                  \vspace*{-1cm}
                  \includestandalone[width=.75\linewidth,keepaspectratio]{figures/disp_gold}
                  \label{fig:disp_gold}
                  % \caption{Conductivity of Gallium Arsenide (GaAs)}
                \end{figure}
              \end{column}
            \end{columns}
          \end{frame}
          \end{document}
